\documentclass[conference]{IEEEtran}

\usepackage[spanish]{babel}
\usepackage{amsmath,amssymb,amsfonts,amsthm}
\usepackage{graphicx}
\usepackage[utf8]{inputenc} % Caracteres en Español (Acentos, ñs)
\usepackage{url} % ACENTOS
\usepackage{hyperref} % Referencias
\usepackage{subfig}
\usepackage{lipsum}
\usepackage{balance} 
\usepackage{etoolbox}
\makeatletter
\patchcmd{\frontmatter@RRAP@format}{(}{}{}{}
\patchcmd{\frontmatter@RRAP@format}{)}{}{}{}
\makeatother	

\usepackage[backend=bibtex,sorting=none]{biblatex}
\setcounter{biburllcpenalty}{7000}
\setcounter{biburlucpenalty}{8000}
\addbibresource{references.bib}

% fecha
\usepackage{datetime}
\newdateformat{specialdate}{
    \twodigit{\THEDAY}-\twodigit{\THEMONTH}-\THEYEAR
}
\date{\specialdate\today}

% la sentencia \burl en las citas... 
\usepackage[hyphenbreaks]{breakurl}
\renewcommand\spanishtablename{Tabla}
\renewcommand\spanishfigurename{Figura}


\begin{document}
% Definitions
\newcommand{\breite}{0.9} %  for twocolumn
\newcommand{\RelacionFiguradoscolumnas}{0.9}
\newcommand{\RelacionFiguradoscolumnasPuntoCinco}{0.45}

%Title of paper
\title{Reporte de Laboratorio 2 \\ Conteo de Frutos Pequeños en Tiempo Real Utilizando una Plataforma Giratoria}

% Trabajo Individual
\author{
    \IEEEauthorblockN{
        Ricardo Emmanuel Uriegas Ibarra\IEEEauthorrefmark{1}
        }
    % En caso de trabajos en equipo, poner a todos los autores 
    % en estricto ORDEN ALFABETICO
    %\author{\IEEEauthorblockN{Michael Shell\IEEEauthorrefmark{1},
    %Homer Simpson\IEEEauthorrefmark{1}}
    \IEEEauthorblockA{
        \IEEEauthorrefmark{1}Ingeniería en Tecnologías de la Información\\
        Universidad Politécnica de Victoria
    }
}

\maketitle

%%%%%%%%%%%%%%%%%%%%%%%%%%%%%%%%%%%%%%%%%%%%%%%%%%%%%%%%%%%%%%%%%%%%%%%
\begin{abstract} 
    
\end{abstract}

%%%%%%%%%%%%%%%%%%%%%%%%%%%%%%%%%%%%%%%%%%%%%%%%%%%%%%%%%%%%%%%%%%%%%%%
\section{Introducción}
La disponibilidad de metodologías prácticas y fiables de detección de frutos en campo resulta fundamental para realizar previsiones precisas de la cosecha\cite{}. 


%%%%%%%%%%%%%%%%%%%%%%%%%%%%%%%%%%%%%%%%%%%%%%%%%%%%%%%%%%%%%%%%%%%%%%%
\section{Desarrollo Experimental}
En este trabajo se presenta un sistema de deteccion en tiempo real de frutos pequeños que giran en una plataforma giratoria.

Para lograr esto se necesita conocimiento en las siguientes áreas:
\begin{itemize}
    \item Procesamiento de Imágenes
    \item Visión por Computadora
    \item Redes Neuronales
\end{itemize}

La mayor parte de estos conocimientos los hemos aprendido en unidades previas de la materia, por lo que no se abordarán en este reporte.

\subsection{Dataset}
Antes de todo debemos definir que es un dataset y porque es que lo ocupamos en este trabajo.
Un dataset es un conjunto de datos que se utilizan para entrenar un modelo de machine learning. En este caso, el dataset que se utilizó fue la union de 2 datasets.

\begin{itemize}
    \item El primer dataset fue recolectado del universo de dataset en roboflow.com\cite{}
    \item El segundo dataset fue recolectado por el mismo equipo del trabajo.
\end{itemize}

\subsection{Modelo de Machine Learning}
Debido a la baja cantidad de imagenes para entrenamiento que poseiamos, sumado a la eficiencia que debia de poseer el modelo para ser ejecutado en tiempo real, se opto por usar el modelo de la libreria ultralytics\cite{} conocido como yolov8n\cite{}.

\subsection{Entrenamiento}
Una vez ya se tenia la seleccion del modelo pre-entrenado a usar, prosiguio con el entrenamiento del modelo con el dataset recolectado.

Para el entrenamiento se realiza lo siguiente:
\begin{itemize}
    \item Se divide el dataset en 3 partes, una para entrenamiento, otra para validacion y otra para pruebas (80\% para entrenamiento, 10\% para validacion y 10\% para pruebas).
    \item Se realiza el entrenamiento del modelo.
    \item Despues se realiza la validacion del modelo.
    \item Y por ultimo; haciendo uso de la seccion del dataset para pruebas, se realiza la prueba a cada epoca modelo y al final se escoge el mejor modelo.
\end{itemize}

%%%%%%%%%%%%%%%%%%%%%%%%%%%%%%%%%%%%%%%%%%%%%%%%%%%%%%%%%%%%%%%%%%%%%%%
\section{Resultados}


%%%%%%%%%%%%%%%%%%%%%%%%%%%%%%%%%%%%%%%%%%%%%%%%%%%%%%%%%%%%%%%%%%%%%%%
\section{Conclusión}


%%%%%%%%%%%%%%%%%%%%%%%%%%%%%%%%%%%%%%%%%%%%%%%%%%%%%%%%%%%%%%%%%%%%%%%
\nocite{calcularRangos}
\addcontentsline{toc}{section}{Referencias} 
\printbibliography
%\balance

\end{document}