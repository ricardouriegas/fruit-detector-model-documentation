\documentclass[conference]{IEEEtran}

\usepackage[spanish]{babel}
\usepackage{amsmath,amssymb,amsfonts,amsthm}
\usepackage{graphicx}
\usepackage[utf8]{inputenc} % Caracteres en Español (Acentos, ñs)
\usepackage{url} % ACENTOS
\usepackage{hyperref} % Referencias
\usepackage{subfig}
\usepackage{lipsum}
\usepackage{balance} 
\usepackage{etoolbox}
\makeatletter
\patchcmd{\frontmatter@RRAP@format}{(}{}{}{}
\patchcmd{\frontmatter@RRAP@format}{)}{}{}{}
\makeatother	

\usepackage[backend=bibtex,sorting=none]{biblatex}
\setcounter{biburllcpenalty}{7000}
\setcounter{biburlucpenalty}{8000}
\addbibresource{references.bib}

% fecha
\usepackage{datetime}
\newdateformat{specialdate}{
    \twodigit{\THEDAY}-\twodigit{\THEMONTH}-\THEYEAR
}
\date{\specialdate\today}

% la sentencia \burl en las citas... 
\usepackage[hyphenbreaks]{breakurl}
\renewcommand\spanishtablename{Tabla}
\renewcommand\spanishfigurename{Figura}


\begin{document}
% Definitions
\newcommand{\breite}{0.9} %  for twocolumn
\newcommand{\RelacionFiguradoscolumnas}{0.9}
\newcommand{\RelacionFiguradoscolumnasPuntoCinco}{0.45}

%Title of paper
\title{Reporte de Laboratorio 2 \\ Conteo de Frutos Pequeños en Tiempo Real Utilizando una Plataforma Giratoria}

% Trabajo Individual
\author{
    \IEEEauthorblockN{
        Ricardo Emmanuel Uriegas Ibarra\IEEEauthorrefmark{1}
        }
    % En caso de trabajos en equipo, poner a todos los autores 
    % en estricto ORDEN ALFABETICO
    %\author{\IEEEauthorblockN{Michael Shell\IEEEauthorrefmark{1},
    %Homer Simpson\IEEEauthorrefmark{1}}
    \IEEEauthorblockA{
        \IEEEauthorrefmark{1}Ingeniería en Tecnologías de la Información\\
        Universidad Politécnica de Victoria
    }
}

\maketitle

%%%%%%%%%%%%%%%%%%%%%%%%%%%%%%%%%%%%%%%%%%%%%%%%%%%%%%%%%%%%%%%%%%%%%%%
\begin{abstract} 
    
\end{abstract}

%%%%%%%%%%%%%%%%%%%%%%%%%%%%%%%%%%%%%%%%%%%%%%%%%%%%%%%%%%%%%%%%%%%%%%%
\section{Introducción}
La disponibilidad de metodologías prácticas y fiables de detección de frutos en campo resulta fundamental para realizar previsiones precisas de la cosecha\cite{}. 


%%%%%%%%%%%%%%%%%%%%%%%%%%%%%%%%%%%%%%%%%%%%%%%%%%%%%%%%%%%%%%%%%%%%%%%
\section{Desarrollo Experimental}
En este trabajo se presenta un sistema de detección en tiempo real de frutos pequeños que giran en una plataforma giratoria.

Para lograr esto se necesita conocimiento en las siguientes áreas:
\begin{itemize}
    \item Procesamiento de Imágenes
    \item Visión por Computadora
    \item Redes Neuronales
\end{itemize}

La mayor parte de estos conocimientos los hemos aprendido en unidades previas de la materia, por lo que no se abordarán en este reporte.

\subsection{Dataset}
Antes de todo debemos definir que es un dataset y porque es que lo ocupamos en este trabajo.
Un dataset es un conjunto de datos que se utilizan para entrenar un modelo de machine learning. En este caso, el dataset que se utilizó fue completamente recolectado por nosotros mismos\cite{}, el cual consiste en imágenes de frutos pequeños en diferentes entornos; el dataset se compone de 1 sola clase, la cual es la de frutos. La cantidad de imágenes recolectada fue de 38 imágenes, pero realizando aumentacion de datos se logro aumentar a 114 imágenes guardadas en roboflow\cite{} para su uso en el entrenamiento del modelo.

Las técnicas de augmentation que se aplicaron a los datos fueron las siguientes:
\begin{itemize}
    \item Flip: Horizontal, Vertical
    \item Rotación 90°: En sentido horario, antihorario, al revés
    \item Crop: 0\% mínimo, 20\% máximo
    \item Rotación: Entre -15° y +15°
    \item Shear: ±10° Horizontal, ±10° Vertical
    \item Hue: Entre -15° y +15°
    \item Saturación: Entre -25\% y +25\%
    \item Brillo: Entre -15\% y +15\%
    \item Exposición: Entre -10\% y +10\%
    \item Blur: Hasta 2.5px
    \item Ruido: Hasta 0.1\% de píxeles
\end{itemize}

\subsection{Modelo}
Debido a la baja cantidad de imágenes para entrenamiento que poseíamos, sumado a la eficiencia que debía de poseer el modelo para ser ejecutado en tiempo real, se opto por usar el modelo pre-entrenando para detección de objetos de la librería ultralytics\cite{} yolov8n\cite{}.

\subsection{Entrenamiento}
Una vez ya se tenia la selección del modelo pre-entrenado a usar, se prosiguió con el entrenamiento del modelo dado el dataset mencionado anteriormente\cite{}.

Para el entrenamiento se realiza lo siguiente:
\begin{itemize}
    \item Se divide el dataset en 3 partes, una para entrenamiento, otra para validación y otra para pruebas (80\% para entrenamiento, 10\% para validación y 10\% para pruebas).
    \item Se realiza el entrenamiento del modelo.
    \item Después se realiza la validación del modelo.
    \item Y por ultimo; haciendo uso de la sección del dataset para pruebas, se realiza la prueba a cada época modelo y al final se escoge el mejor modelo.
\end{itemize}

\subsection{Uso del Modelo}
Para usar el modelo en el código se agrego un archivo (UI.py) que haciendo uso de una interfaz gráfica en Qt6, carga el modelo entrenado y realiza la detección de los frutos en tiempo real mediante la cámara de la computadora.

%%%%%%%%%%%%%%%%%%%%%%%%%%%%%%%%%%%%%%%%%%%%%%%%%%%%%%%%%%%%%%%%%%%%%%%
\section{Resultados}



%%%%%%%%%%%%%%%%%%%%%%%%%%%%%%%%%%%%%%%%%%%%%%%%%%%%%%%%%%%%%%%%%%%%%%%
\section{Conclusión}


%%%%%%%%%%%%%%%%%%%%%%%%%%%%%%%%%%%%%%%%%%%%%%%%%%%%%%%%%%%%%%%%%%%%%%%
\nocite{calcularRangos}
\addcontentsline{toc}{section}{Referencias} 
\printbibliography
%\balance

\end{document}